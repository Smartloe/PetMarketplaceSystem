%% chapter-2:引言

\chapter{引言}
\section{研究背景与目的}

随着我国宠物养育数量的不断增加和宠物市场的多元化发展,宠物行业正在经历着快速增长的阶段。根据亚宠研究院发布的《宠物行业蓝皮书-2023宠物行业发展报告》,2018-2022年我国的养宠数量持续上升,养宠市场规模不断扩大,成为一个高速发展的行业。该报告预测,到2023年,我国养宠数量将达到近2.0亿只,市场规模将突破2500亿元。\par  % 段落分隔
宠物市场的发展呈现出多元化的趋势,消费者洞察显示,养宠人群主要以90后为主,他们更倾向于通过线上渠道购买宠物用品和服务。这一趋势的崛起在一定程度上推动了宠物行业电商化的发展,为市场注入了活力。然而,一些宠物商店仍然只专注于线下零售渠道,可能由于技术水平的不足、资金有限或者观望市场趋势而错失了在线市场的机会,因为越来越多的消费者选择在网上购物。\par  % 段落分隔
“十三五”时期,我国电子商务取得了显著成就:电子商务交易额从2015年的21.8万亿元增至2020年的37.2万亿元;全国网上零售额2020年达到11.8万亿元,我国已连续8年成为全球规模最大的网络零售市场;2020年实物商品网上零售额占社会消费品零售总额的比重接近四分之一,电子商务已经成为居民消费的主渠道之一;电子商务从业人员规模超过6000万,电商新业态、新模式创造了大量新职业、新岗位,成为重要的“社会稳定器”。这些数据充分说明,电子商务已经全面融入我国生产生活各领域,成为提升人民生活品质和推动经济社会发展的重要力量。与此同时,我国电子商务发展仍然面临不规范、不充分、不平衡的问题,平台企业垄断和不公平竞争问题凸显,企业核心竞争力不强,外部宏观环境发生复杂深刻变化,电子商务高质量发展机遇和挑战并存。\par  % 段落分隔
党中央、国务院高度关注电子商务的发展,在《“十四五”电子商务发展规划》中明确指出电子商务是大有可为的,并在各个层面提出了促进电子商务发展的要求。特别是在抗击新冠肺炎疫情过程中,电子商务发挥了重要作用,加速了线上线下融合的趋势,为经济发展注入了新的活力。\par  % 段落分隔
在“十四五”规划提出了总体要求,指导思想强调以新发展理念为核心,坚持守正创新、规范发展,推动电子商务与各产业深度融合,通过数字化手段赋能经济社会转型,保障安全健康发展,实现普惠共享与开放共赢的目标。\par  % 段落分隔
在当前宠物市场多元化以及电商行业大有可为的背景下,设计和实现一个宠物在线商城是一个具有潜力的创新举措。通过搭建宠物商城平台,不仅可以更好地满足广泛养宠人群的需求,提供全方位的宠物商品和服务,还有助于增加商家的盈收并提高管理效率。这样的举措不仅能够顺应市场趋势,也为宠物行业的未来发展打开了新的可能性。\par  % 段落分隔
本课题旨在通过基于Django的宠物商城设计与实现,满足宠物主人对购物的便捷需求,提高宠物商家的经营效益。具体目的和意义如下:\par  % 段落分隔
(1)提升用户体验:通过设计直观、易用的UI界面,提供便捷的商品浏览、购买、支付等功能,提升用户在宠物商城的购物体验。\par  % 段落分隔
(2)优化商家管理:提供商家端的管理系统,包括商品管理、订单管理、库存管理等功能,协助商家更高效地运营宠物商城。\par  % 段落分隔
(3)提升市场竞争力:在宠物用品市场日益竞争激烈的情况下,通过提供简便易用的宠物商城平台,帮助小型宠物商家在市场中更好地立足和竞争。\par  % 段落分隔
(4)促进本地宠物社群互动:通过商城中的评论、社交分享等功能,促进宠物主人之间的互动,形成本地宠物社群,增进用户黏性。\par  % 段落分隔
\section{国内外研究现状}
\subsection{国外研究现状}

\subsection{国内研究现状}

\section{研究主要内容}

\section{研究方法}
\subsection{文献研究法}

\subsection{系统分析与设计}

\subsection{技术选型与实现}

\section{论文组织架构(技术路线图)}

